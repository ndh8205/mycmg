\documentclass{article}
\usepackage{graphicx}
\usepackage{kotex}
\usepackage{amsmath}
\usepackage{amssymb}
\usepackage{graphicx}
\usepackage{subfigure}
\usepackage{multirow}
\usepackage{geometry}
\usepackage{booktabs}

\geometry{a4paper, margin=2.5cm}

\title{\textbf{GAiMSat-1 프로젝트\\Nomenclature \& 수학적 표기법}}
\author{남동현}
\date{November 2025}

\begin{document}

\maketitle

\section{기본 표기 규칙}

\subsection{스칼라, 벡터, 행렬}
\begin{itemize}
    \item \textbf{스칼라}: 굵지 않은 기호 ($a, A, \gamma, \Gamma$)
    \item \textbf{벡터}: 굵은 소문자 ($\mathbf{a}, \boldsymbol{\gamma}$)
    \item \textbf{행렬}: 굵은 대문자 ($\mathbf{A}, \boldsymbol{\Gamma}$)
\end{itemize}

\subsection{좌표계 및 프레임 표현}
좌표계(프레임)는 필기체 대문자 $\mathcal{I}, \mathcal{A}, \mathcal{B}, \mathcal{L}$ 등으로 표기한다.

\textbf{표기 규칙:}
\begin{itemize}
    \item 프레임 $\mathcal{I}$에서 표현된 점 $P$의 위치 벡터: $\mathbf{r}_P^{\mathcal{I}}$
    \item 프레임 $\mathcal{A}$에서 표현된, 프레임 $\mathcal{I}$에 대한 프레임 $\mathcal{A}$의 각속도: $\boldsymbol{\omega}_{\mathcal{A}/\mathcal{I}}^{\mathcal{A}}$
\end{itemize}

\begin{table}[h]
\centering
\begin{tabular}{cl}
\toprule
\textbf{기호} & \textbf{설명} \\
\midrule
$\mathcal{I}$ & Inertial frame (ECI) \\
$\mathcal{A}$ & Chief body frame \\
$\mathcal{B}$ & Deputy body frame \\
$\mathcal{L}$ & LVLH (Local Vertical Local Horizontal) frame \\
\bottomrule
\end{tabular}
\caption{본 프로젝트에서 사용되는 주요 좌표계}
\end{table}

\subsection{회전 표현}
프레임 $\mathcal{A}$에서 프레임 $\mathcal{I}$로의 회전:
\begin{itemize}
    \item 회전행렬: $\mathbf{R}_{\mathcal{A}}^{\mathcal{I}}$ (프레임 $\mathcal{A}$의 벡터를 프레임 $\mathcal{I}$로 변환)
    \item 쿼터니언: $\mathbf{q}_{\mathcal{A}}^{\mathcal{I}}$
\end{itemize}

\section{벡터 외적 연산자}

\subsection{반대칭 행렬 (Skew-Symmetric Matrix)}
벡터 $\mathbf{a} = [a_1 \quad a_2 \quad a_3]^T$에 대한 반대칭 행렬:
\begin{equation}
[\mathbf{a}]_{\times} = \begin{bmatrix} 0 & -a_3 & a_2 \\ a_3 & 0 & -a_1 \\ -a_2 & a_1 & 0 \end{bmatrix}
\end{equation}

이 행렬은 외적 연산을 행렬 곱셈으로 표현: $\mathbf{a} \times \mathbf{b} = [\mathbf{a}]_{\times}\mathbf{b}$

\subsection{주요 성질}
\begin{enumerate}
    \item 반대칭성: $[\mathbf{a}]_{\times}^T = -[\mathbf{a}]_{\times}$
    \item 대각합: $\text{tr}([\mathbf{a}]_{\times}) = 0$
    \item 쌍대성: $[\mathbf{a}]_{\times}\mathbf{b} = -[\mathbf{b}]_{\times}\mathbf{a}$
    \item 벡터 삼중곱: $\mathbf{a} \times (\mathbf{b} \times \mathbf{c}) = [\mathbf{a}]_{\times}[\mathbf{b}]_{\times}\mathbf{c}$
\end{enumerate}

\subsection{회전 동역학에서의 응용}
회전행렬의 시간 미분:
\begin{equation}
\dot{\mathbf{R}}_{\mathcal{A}}^{\mathcal{I}} = 
\begin{cases}
\mathbf{R}_{\mathcal{A}}^{\mathcal{I}}[\boldsymbol{\omega}_{\mathcal{A}/\mathcal{I}}^{\mathcal{A}}]_{\times} & \text{(각속도를 $\mathcal{A}$ 프레임에서 표현 - Local 표현 미분)} \\
[\boldsymbol{\omega}_{\mathcal{A}/\mathcal{I}}^{\mathcal{I}}]_{\times}\mathbf{R}_{\mathcal{A}}^{\mathcal{I}} & \text{(각속도를 $\mathcal{I}$ 프레임에서 표현 - Global 표현 미분)}
\end{cases}
\end{equation}

body frame을 기준으로 자세(회전)를 표현 했을때의 각속도를 매칭하게 되면 $\dot{\mathbf{R}}_{\mathcal{B}}^{\mathcal{I}} = \mathbf{R}_{\mathcal{B}}^{\mathcal{I}}[\boldsymbol{\omega}_{\mathcal{B}/\mathcal{I}}^{\mathcal{B}}]_{\times}$의 형태가 일반적이다.

\section{쿼터니언 표현}

\subsection{기본 정의와 Hamilton Convention}
본 프로젝트에서는 \textbf{Hamiltonian convention (Scalar-first)}을 사용한다:
\begin{equation}
\mathbf{q} = [q_w \quad \mathbf{q}_v^T]^T = [q_w \quad q_x \quad q_y \quad q_z]^T, \quad \|\mathbf{q}\| = 1
\end{equation}
여기서 $q_w$는 스칼라 부분, $\mathbf{q}_v = [q_x \quad q_y \quad q_z]^T$는 벡터 부분이다.

\vspace{1em}

\textbf{중요:} Hamilton convention에서 쿼터니언 $\mathbf{q}_{\mathcal{A}}^{\mathcal{I}}$의 아래첨자는 항상 local frame(회전 전 프레임)을 나타내고, 위첨자는 global frame(회전 후 프레임)을 나타낸다. 즉, $\mathbf{q}_{\mathcal{A}}^{\mathcal{I}}$는 프레임 $\mathcal{A}$에서 프레임 $\mathcal{I}$로의 회전을 표현한다. 반대의 경우 Local과 Global의 정의가 바뀌는 것이 아니다. $\mathbf{q}_{\mathcal{I}}^{\mathcal{A}}$의 경우에는 아래첨자 ${\mathcal{I}}$가 local frame(회전 전 프레임)을 나타내고, 위첨자 ${\mathcal{A}}$가  global frame(회전 후 프레임)을 나타낸다.


\subsection{기본 연산}
\begin{table}[h]
\centering
\begin{tabular}{ll}
\toprule
\textbf{연산} & \textbf{표현} \\
\midrule
켤레 (Conjugate) & $\mathbf{q}^* = [q_w \quad -\mathbf{q}_v^T]^T$ \\

\medskip

역원 (Inverse) & $\mathbf{q}^{-1} = \frac{\mathbf{q}^*}{\|\mathbf{q}\|^2} = \mathbf{q}^*$ (단위 쿼터니언) \\

\medskip

쿼터니언 곱셈 & $\mathbf{q}_{\mathcal{A}}^{\mathcal{I}} = \mathbf{q}_{\mathcal{B}}^{\mathcal{I}} \otimes \mathbf{q}_{\mathcal{A}}^{\mathcal{B}}$ \\
\bottomrule
\end{tabular}
\end{table}

\subsection{쿼터니언 곱셈}
두 쿼터니언의 곱셈:
\begin{equation}
\mathbf{p} = \mathbf{q}_1 \otimes \mathbf{q}_2 = [\mathbf{q}_1]_L \mathbf{q}_2 = [\mathbf{q}_2]_R \mathbf{q}_1
\end{equation}

임의의 쿼터니언 $\mathbf{q} = [q_w \quad \mathbf{q}_v^T]^T$에 대해 좌곱 및 우곱 행렬을 다음과 같이 정의할 수 있다:
\begin{equation}
\boldsymbol{\Gamma}(\mathbf{q}) = [\mathbf{q}]_L = \begin{bmatrix} 
q_w & -\mathbf{q}_v^T \\
\mathbf{q}_v & q_w\mathbf{I}_3 + [\mathbf{q}_v]_{\times}
\end{bmatrix}, \quad
\boldsymbol{\Omega}(\mathbf{q}) = [\mathbf{q}]_R = \begin{bmatrix}
q_w & -\mathbf{q}_v^T \\
\mathbf{q}_v & q_w\mathbf{I}_3 - [\mathbf{q}_v]_{\times}
\end{bmatrix}
\end{equation}

\subsection{보조 행렬}
\begin{equation}
\boldsymbol{\Xi}(\mathbf{q}) = \begin{bmatrix}
-\mathbf{q}_v^T \\
q_w\mathbf{I}_3 + [\mathbf{q}_v]_{\times}
\end{bmatrix}, \quad
\boldsymbol{\Psi}(\mathbf{q}) = \begin{bmatrix}
-\mathbf{q}_v^T \\
q_w\mathbf{I}_3 - [\mathbf{q}_v]_{\times}
\end{bmatrix}
\end{equation}

$\boldsymbol{\Xi}(\mathbf{q})$와 $\boldsymbol{\Psi}(\mathbf{q})$는 각각 $[\mathbf{q}]_L$, $[\mathbf{q}]_R$의 하부 $3 \times 4$ 블록에 해당한다.


\subsection{변환 공식}
쿼터니언에서 회전행렬로의 변환은 보조행렬의 연산으로 표현할 수 있음:
\begin{equation}
\mathbf{R} = \boldsymbol{\Psi}(\mathbf{q})^T\boldsymbol{\Xi}(\mathbf{q})
\end{equation}
이러한 정의로 유도한 회전행렬은 쿼터니언이 나타내는 회전을 그대로 표현하여 나타낼 수 있음:
\begin{equation}
\mathbf{R}_{\mathcal{A}}^{\mathcal{L}} = \boldsymbol{\Psi}(\mathbf{q}_{\mathcal{A}}^{\mathcal{L}})^T\boldsymbol{\Xi}(\mathbf{q}_{\mathcal{A}}^{\mathcal{L}})
\end{equation}


특히, 순수 쿼터니언(pure quaternion) $\mathbf{p} = [0 \quad \boldsymbol{\omega}^T]^T$의 경우 $q_w = 0$이고 $\mathbf{q}_v = \boldsymbol{\omega}$이므로:
\begin{equation}
\boldsymbol{\Gamma}(\boldsymbol{\omega}) = \begin{bmatrix} 
0 & -\boldsymbol{\omega}^T \\
\boldsymbol{\omega} & [\boldsymbol{\omega}]_{\times}
\end{bmatrix}, \quad
\boldsymbol{\Omega}(\boldsymbol{\omega}) = \begin{bmatrix}
0 & -\boldsymbol{\omega}^T \\
\boldsymbol{\omega} & -[\boldsymbol{\omega}]_{\times}
\end{bmatrix}
\end{equation}

이러한 행렬들은 쿼터니언 동역학 방정식에서 각속도를 쿼터니언 미분으로 변환하는 데 사용된다.

\section{쿼터니언 동역학}

\subsection{단일 프레임의 미분 (Absolute Quaternion Kinematics)}
프레임 $\mathcal{A}$의 관성좌표계 $\mathcal{I}$에 대한 쿼터니언 시간 미분:
\begin{equation}
\dot{\mathbf{q}}_{\mathcal{A}}^{\mathcal{I}} = \frac{1}{2}\mathbf{q}_{\mathcal{A}}^{\mathcal{I}}\otimes\begin{bmatrix} 0 \\ \boldsymbol{\omega}_{\mathcal{A}/\mathcal{I}}^{\mathcal{A}} \end{bmatrix}
= \frac{1}{2}\boldsymbol{\Omega}(\boldsymbol{\omega}_{\mathcal{A}/\mathcal{I}}^{\mathcal{A}})\mathbf{q}_{\mathcal{A}}^{\mathcal{I}}= \frac{1}{2}\boldsymbol{\Gamma}(\mathbf{q}_{\mathcal{A}}^{\mathcal{I}})\begin{bmatrix} 0 \\ \boldsymbol{\omega}_{\mathcal{A}/\mathcal{I}}^{\mathcal{A}} \end{bmatrix}
\end{equation}

여기서 각속도 는 local frame $\mathcal{A}$에서 표현된 $\boldsymbol{\omega}_{\mathcal{A}/\mathcal{I}}^{\mathcal{A}}$을 사용한다.

\subsection{상대 쿼터니언의 동역학 (Relative Quaternion Kinematics)}

\subsubsection{상대 쿼터니언의 정의}
두 body frame $\mathcal{A}$ (Chief)와 $\mathcal{B}$ (Deputy) 사이의 상대 자세를 나타내는 쿼터니언 $\mathbf{q}_{\mathcal{B}}^{\mathcal{A}}$는 다음과 같이 정의된다:
\begin{equation}
\mathbf{q}_{\mathcal{B}}^{\mathcal{I}} = \mathbf{q}_{\mathcal{A}}^{\mathcal{I}} \otimes \mathbf{q}_{\mathcal{B}}^{\mathcal{A}}
\end{equation}

\begin{equation}
\mathbf{q}_{\mathcal{B}}^{\mathcal{A}} = (\mathbf{q}_{\mathcal{A}}^{\mathcal{I}})^{-1}  \otimes \mathbf{q}_{\mathcal{B}}^{\mathcal{I}}
\end{equation}

물리적으로 $\mathbf{q}_{\mathcal{B}}^{\mathcal{A}}$는 Chief의 body frame $\mathcal{A}$에서 본 Deputy의 body frame $\mathcal{B}$의 자세를 나타낸다.

\subsubsection{일반적인 상대 쿼터니언 미분}
프레임 $\mathcal{A}$의 관성좌표계 $\mathcal{I}$에 대한 쿼터니언 시간 미분:
\begin{equation}
\dot{\mathbf{q}}_{\mathcal{B}}^{\mathcal{A}} = \frac{1}{2}\mathbf{q}_{\mathcal{B}}^{\mathcal{A}}\otimes\begin{bmatrix} 0 \\ \boldsymbol{\omega}_{\mathcal{A}/\mathcal{B}}^{\mathcal{A}} \end{bmatrix}
= \frac{1}{2}\boldsymbol{\Omega}(\boldsymbol{\omega}_{\mathcal{A}/\mathcal{B}}^{\mathcal{A}})\mathbf{q}_{\mathcal{B}}^{\mathcal{A}}= \frac{1}{2}\boldsymbol{\Gamma}(\mathbf{q}_{\mathcal{B}}^{\mathcal{A}})\begin{bmatrix} 0 \\ \boldsymbol{\omega}_{\mathcal{A}/\mathcal{B}}^{\mathcal{A}} \end{bmatrix}
\end{equation}

이와 같이 유도된 일반적인 상대 쿼터니언의 미분은 카메라로 혹은 상대적인 위치에서 관측할 수 있는 센서로 측정되는 되는 각속도 $\boldsymbol{\omega}_{\mathcal{A}/\mathcal{B}}^{\mathcal{A}}$ 를 이용한다.

\subsubsection{협조 케이스의 상대 쿼터니언 미분의 유도}
상대 쿼터니언의 정의로부터:
\begin{equation}
\mathbf{q}_{\mathcal{B}}^{\mathcal{A}} = \left((\mathbf{q}_{\mathcal{A}}^{\mathcal{I}})^{-1} \otimes \mathbf{q}_{\mathcal{B}}^{\mathcal{I}}\right) = \left(\mathbf{q}_{\mathcal{I}}^{\mathcal{A}} \otimes \mathbf{q}_{\mathcal{B}}^{\mathcal{I}}\right)
\end{equation}

시간 미분하면:
\begin{equation}
\dot{\mathbf{q}}_{\mathcal{B}}^{\mathcal{A}} = \dot{\mathbf{q}}_{\mathcal{I}}^{\mathcal{A}} \otimes \mathbf{q}_{\mathcal{B}}^{\mathcal{I}} + \mathbf{q}_{\mathcal{I}}^{\mathcal{A}} \otimes \dot{\mathbf{q}}_{\mathcal{B}}^{\mathcal{I}}
\end{equation}

각 절대 쿼터니언 미분에 동역학 방정식을 대입:
\begin{align}
\dot{\mathbf{q}}_{\mathcal{B}}^{\mathcal{A}} &= \frac{1}{2}\begin{bmatrix} 0 \\ \boldsymbol{\omega}_{\mathcal{I}/\mathcal{A}}^{\mathcal{A}} \end{bmatrix} \otimes \mathbf{q}_{\mathcal{I}}^{\mathcal{A}} \otimes \mathbf{q}_{\mathcal{B}}^{\mathcal{I}} + \mathbf{q}_{\mathcal{I}}^{\mathcal{A}} \otimes \frac{1}{2}\mathbf{q}_{\mathcal{B}}^{\mathcal{I}} \otimes \begin{bmatrix} 0 \\ \boldsymbol{\omega}_{\mathcal{B}/\mathcal{I}}^{\mathcal{B}} \end{bmatrix} \\
\end{align}

곱셈 행렬 형태로 변환하고 $\mathbf{q}_{\mathcal{B}}^{\mathcal{A}} = \mathbf{q}_{\mathcal{I}}^{\mathcal{A}} \otimes \mathbf{q}_{\mathcal{B}}^{\mathcal{I}}$를 이용:
\begin{equation}
= \frac{1}{2}\boldsymbol{\Gamma}(\boldsymbol{\omega}_{\mathcal{I}/\mathcal{A}}^{\mathcal{A}})\mathbf{q}_{\mathcal{B}}^{\mathcal{A}} + \frac{1}{2}\boldsymbol{\Omega}(\boldsymbol{\omega}_{\mathcal{B}/\mathcal{I}}^{\mathcal{B}})\mathbf{q}_{\mathcal{B}}^{\mathcal{A}}
\end{equation}

각속도 부호 변환 ($\boldsymbol{\omega}_{\mathcal{I}/\mathcal{A}}^{\mathcal{A}} = -\boldsymbol{\omega}_{\mathcal{A}/\mathcal{I}}^{\mathcal{A}}$):

\begin{equation}
= \frac{1}{2}\boldsymbol{\Omega}(\boldsymbol{\omega}_{\mathcal{B}/\mathcal{I}}^{\mathcal{B}})\mathbf{q}_{\mathcal{B}}^{\mathcal{A}} - \frac{1}{2}\boldsymbol{\Gamma}(\boldsymbol{\omega}_{\mathcal{A}/\mathcal{I}}^{\mathcal{A}})\mathbf{q}_{\mathcal{B}}^{\mathcal{A}}
\end{equation}

인수분해:
\begin{equation}
= \frac{1}{2}\left(\boldsymbol{\Omega}(\boldsymbol{\omega}_{\mathcal{B}/\mathcal{I}}^{\mathcal{B}}) - \frac{1}{2}\boldsymbol{\Gamma}(\boldsymbol{\omega}_{\mathcal{A}/\mathcal{I}}^{\mathcal{A}})\right)\mathbf{q}_{\mathcal{B}}^{\mathcal{A}}
\end{equation}

최종적으로:
\begin{equation}
\dot{\mathbf{q}}_{\mathcal{B}}^{\mathcal{A}} = \frac{1}{2}\boldsymbol{\Theta}(\boldsymbol{\omega}_{\mathcal{A}/\mathcal{I}}^{\mathcal{A}}, \boldsymbol{\omega}_{\mathcal{B}/\mathcal{I}}^{\mathcal{B}})\mathbf{q}_{\mathcal{B}}^{\mathcal{A}}
\end{equation}

여기서:
\begin{equation}
\boldsymbol{\Theta}(\boldsymbol{\omega}_{\mathcal{A}/\mathcal{I}}^{\mathcal{A}}, \boldsymbol{\omega}_{\mathcal{B}/\mathcal{I}}^{\mathcal{B}}) = \boldsymbol{\Omega}(\boldsymbol{\omega}_{\mathcal{B}/\mathcal{I}}^{\mathcal{B}}) - \boldsymbol{\Gamma}(\boldsymbol{\omega}_{\mathcal{A}/\mathcal{I}}^{\mathcal{A}})
\end{equation}

\subsubsection{협조케이스의 물리적 해석}
$\boldsymbol{\Theta}$ 행렬은 두 개의 항으로 구성된다:
\begin{itemize}
    \item $\boldsymbol{\Omega}(\boldsymbol{\omega}_{\mathcal{B}/\mathcal{I}}^{\mathcal{B}})$: Deputy의 절대 각속도 기여. Deputy가 관성 공간에서 회전하는 효과.
    \item $-\boldsymbol{\Gamma}(\boldsymbol{\omega}_{\mathcal{A}/\mathcal{I}}^{\mathcal{A}})$: Chief의 절대 각속도 기여. Chief의 회전으로 인해 상대 자세가 변하는 효과를 제거 (음수 부호).
\end{itemize}

이 식은 두 프레임의 절대 각속도로부터 상대 자세의 시간 변화율을 계산한다. 각 각속도는 해당 프레임의 local frame에서 표현된 값을 사용한다. 이는 각 위성의 상태가 공유되고 있을 때 사용하기 매우 용이하다.

\subsubsection{Formation Flying에서의 응용}
Formation flying 시나리오에서 Chief와 Deputy의 자세를 각각 IMU와 Star Tracker로 측정한다고 가정하자:
\begin{itemize}
    \item Chief의 각속도: $\boldsymbol{\omega}_{\mathcal{A}/\mathcal{I}}^{\mathcal{A}}$ (Chief의 body frame에서 측정)
    \item Deputy의 각속도: $\boldsymbol{\omega}_{\mathcal{B}/\mathcal{I}}^{\mathcal{B}}$ (Deputy의 body frame에서 측정)
\end{itemize}

이 두 측정값을 이용하여 상대 자세 $\mathbf{q}_{\mathcal{B}}^{\mathcal{A}}$의 시간 변화를 propagate할 수 있다.

\subsection{각속도 추출}
쿼터니언 미분으로부터 각속도를 추출:
\begin{equation}
\boldsymbol{\omega}_{\mathcal{A}/\mathcal{I}}^{\mathcal{A}} = 2\boldsymbol{\Xi}^T(\mathbf{q}_{\mathcal{A}}^{\mathcal{I}})\dot{\mathbf{q}}_{\mathcal{A}}^{\mathcal{I}}
\end{equation}

\section{쿼터니언 연속성 보장}

쿼터니언 $\mathbf{q}$와 $-\mathbf{q}$는 동일한 회전을 나타낸다. 시계열 데이터에서 연속성을 보장하기 위해 다음 조건을 확인한다:
\begin{equation}
\text{if } \mathbf{q}_k^T\mathbf{q}_{k-1} < 0 \text{, then } \mathbf{q}_k \leftarrow -\mathbf{q}_k
\end{equation}
이를 통해 쿼터니언 궤적이 최단 경로를 따르도록 하고, 불연속적인 점프를 방지한다.


\section{Formation Flying에서의 좌표계 및 각속도 계산}

\subsection{LVLH (Local Vertical Local Horizontal) 프레임}

Chief 위성의 위치 $\mathbf{r}_{\mathcal{A}}^{\mathcal{I}}$와 속도 $\mathbf{v}_{\mathcal{A}}^{\mathcal{I}}$로부터 LVLH 프레임 $\mathcal{L}$을 다음과 같이 정의한다:

\begin{equation}
\mathbf{i}_{\mathcal{L}} = \frac{\mathbf{r}_{\mathcal{A}}^{\mathcal{I}}}{\|\mathbf{r}_{\mathcal{A}}^{\mathcal{I}}\|} \quad \text{(Radial direction)}
\end{equation}

\begin{equation}
\mathbf{h} = \mathbf{r}_{\mathcal{A}}^{\mathcal{I}} \times \mathbf{v}_{\mathcal{A}}^{\mathcal{I}}, \quad
\mathbf{k}_{\mathcal{L}} = \frac{\mathbf{h}}{\|\mathbf{h}\|} \quad \text{(Angular momentum direction)}
\end{equation}

\begin{equation}
\mathbf{j}_{\mathcal{L}} = \mathbf{k}_{\mathcal{L}} \times \mathbf{i}_{\mathcal{L}} \quad \text{(Along-track direction)}
\end{equation}

\vspace{1em}

LVLH에서 ECI로의 회전행렬:
\begin{equation}
\mathbf{R}_{\mathcal{L}}^{\mathcal{I}} = [\mathbf{i}_{\mathcal{L}}, \mathbf{j}_{\mathcal{L}}, \mathbf{k}_{\mathcal{L}}]
\end{equation}

\subsection{LVLH 프레임의 각속도}

LVLH 프레임의 관성좌표계에 대한 각속도 (ECI에서 관찰):
\begin{equation}
\boldsymbol{\omega}_{\mathcal{L}/\mathcal{I}}^{\mathcal{I}} = \frac{\mathbf{r}_{\mathcal{A}}^{\mathcal{I}} \times \mathbf{v}_{\mathcal{A}}^{\mathcal{I}}}{\|\mathbf{r}_{\mathcal{A}}^{\mathcal{I}}\|^2} = \frac{\mathbf{h}}{\|\mathbf{r}_{\mathcal{A}}^{\mathcal{I}}\|^2}
\end{equation}

LVLH 프레임에서 표현:
\begin{equation}
\boldsymbol{\omega}_{\mathcal{L}/\mathcal{I}}^{\mathcal{L}} = \mathbf{R}_{\mathcal{I}}^{\mathcal{L}} \boldsymbol{\omega}_{\mathcal{L}/\mathcal{I}}^{\mathcal{I}}
\end{equation}

\subsubsection{원궤도 근사}

원궤도의 경우, LVLH 각속도는 cross-track 방향 (z축)의 일정한 각속도로 근사할 수 있다:
\begin{equation}
\boldsymbol{\omega}_{\mathcal{L}/\mathcal{I}}^{\mathcal{L}} \approx \begin{bmatrix} 0 \\ 0 \\ n \end{bmatrix}
\end{equation}

여기서 $n$은 평균 운동 (mean motion):
\begin{equation}
n = \sqrt{\frac{\mu}{a^3}} \quad \text{[rad/s]}
\end{equation}

$\mu$는 지구 중력 상수, $a$는 궤도 장반경이다.

\subsubsection{일반 궤도에서의 벡터 형태}

타원 궤도를 포함한 일반적인 경우, LVLH 프레임에서 표현된 각속도는:
\begin{equation}
\boldsymbol{\omega}_{\mathcal{L}/\mathcal{I}}^{\mathcal{L}} = \begin{bmatrix} 0 \\ 0 \\ \frac{h}{\|\mathbf{r}_{\mathcal{A}}^{\mathcal{I}}\|^2} \end{bmatrix}
\end{equation}

여기서:
\begin{itemize}
    \item $h = \|\mathbf{h}\| = \|\mathbf{r}_{\mathcal{A}}^{\mathcal{I}} \times \mathbf{v}_{\mathcal{A}}^{\mathcal{I}}\|$: 비각운동량 (specific angular momentum) [km$^2$/s]
\end{itemize}

원궤도에서는 $h = \sqrt{\mu a}$이고 $\|\mathbf{r}_{\mathcal{A}}^{\mathcal{I}}\| = a$이므로 $\frac{h}{\|\mathbf{r}_{\mathcal{A}}^{\mathcal{I}}\|^2} = \frac{\sqrt{\mu a}}{a^2} = \sqrt{\frac{\mu}{a^3}} = n$이 되어 위의 근사식과 일치한다.

\subsubsection{궤도 주기}

평균 운동과 궤도 주기의 관계:
\begin{equation}
T_{\text{orbit}} = \frac{2\pi}{n} \quad \text{[s]}
\end{equation}

\subsection{Chief 위성 (Frame A)의 각속도}

Chief가 LVLH 프레임에 대해 회전하는 경우, LVLH에서 표현된 Chief의 상대 각속도를 회전축 $\hat{\mathbf{e}}^{\mathcal{L}} = [e_x, e_y, e_z]^T$ (단위벡터)와 각속도 크기 $\omega_{\text{rot}}$로 표현하면:

\begin{equation}
\boldsymbol{\omega}_{\mathcal{A}/\mathcal{L}}^{\mathcal{L}} = \omega_{\text{rot}} \hat{\mathbf{e}}^{\mathcal{L}} = \omega_{\text{rot}} \begin{bmatrix} 
e_x \\ 
e_y \\ 
e_z 
\end{bmatrix}, \quad \|\hat{\mathbf{e}}^{\mathcal{L}}\| = 1
\end{equation}

Chief의 절대 각속도 (Chief frame에서 표현):
\begin{equation}
\boldsymbol{\omega}_{\mathcal{A}/\mathcal{I}}^{\mathcal{A}} = \boldsymbol{\omega}_{\mathcal{A}/\mathcal{L}}^{\mathcal{A}} + \boldsymbol{\omega}_{\mathcal{L}/\mathcal{I}}^{\mathcal{A}}
\end{equation}

여기서:
\begin{align}
\boldsymbol{\omega}_{\mathcal{A}/\mathcal{L}}^{\mathcal{A}} &= -\mathbf{R}_{\mathcal{L}}^{\mathcal{A}} \boldsymbol{\omega}_{\mathcal{A}/\mathcal{L}}^{\mathcal{L}} \\
\boldsymbol{\omega}_{\mathcal{L}/\mathcal{I}}^{\mathcal{A}} &= \mathbf{R}_{\mathcal{L}}^{\mathcal{A}} \boldsymbol{\omega}_{\mathcal{L}/\mathcal{I}}^{\mathcal{L}}
\end{align}

\subsection{Deputy 위성 (Frame B)의 각속도 - No-Roll Pointing}

\subsubsection{No-Roll Pointing 자세 결정}
Deputy는 Chief를 지향하며 roll 회전을 억제하는 자세 제어를 수행한다. 이는 다음과 같이 구현된다:
\begin{enumerate}
    \item \textbf{Pointing 방향}: Deputy의 body Y축($\mathbf{b}_y$)이 Chief를 향함
    \item \textbf{Roll 제약}: 궤도 평면 법선($\mathbf{h}_{orbit}$)을 이용한 roll 억제
\end{enumerate}

Deputy body frame의 축 정의:
\begin{align}
\mathbf{b}_y &= \frac{-\mathbf{r}_{\mathcal{B}}^{\mathcal{L}}}{\|\mathbf{r}_{\mathcal{B}}^{\mathcal{L}}\|} \quad \text{(Chief pointing)} \\
\mathbf{b}_x &= \frac{\mathbf{b}_y \times \mathbf{h}_{orbit}}{\|\mathbf{b}_y \times \mathbf{h}_{orbit}\|} \quad \text{(No-roll constraint)} \\
\mathbf{b}_z &= \mathbf{b}_x \times \mathbf{b}_y \quad \text{(Right-handed system)}
\end{align}

이로부터 회전행렬을 계산:
\begin{equation}
\mathbf{R}_{\mathcal{L}}^{\mathcal{B}} = [\mathbf{b}_x, \mathbf{b}_y, \mathbf{b}_z]^T
\end{equation}

여기서 $\mathbf{h}_{orbit}$은 GCO Formation에서 Deputy가 만들어내는 원 궤도의 각속도다. 

\subsubsection{Deputy 각속도 계산}
Deputy가 Chief를 향해 no-roll pointing을 수행하는 경우, Deputy의 LVLH에 대한 각속도는 상대 위치 벡터 $\mathbf{r}_{\mathcal{B}}^{\mathcal{L}}$의 회전 각속도와 같다:

\begin{equation}
\boldsymbol{\omega}_{\mathcal{B}/\mathcal{L}}^{\mathcal{L}} = \frac{\mathbf{r}_{\mathcal{B}}^{\mathcal{L}} \times \mathbf{v}_{\mathcal{B}}^{\mathcal{L}}}{\|\mathbf{r}_{\mathcal{B}}^{\mathcal{L}}\|^2}
\end{equation}

Deputy의 절대 각속도 (Deputy frame에서 표현):
\begin{equation}
\boldsymbol{\omega}_{\mathcal{B}/\mathcal{I}}^{\mathcal{B}} = \boldsymbol{\omega}_{\mathcal{B}/\mathcal{L}}^{\mathcal{B}} + \boldsymbol{\omega}_{\mathcal{L}/\mathcal{I}}^{\mathcal{B}}
\end{equation}

여기서:
\begin{align}
\boldsymbol{\omega}_{\mathcal{B}/\mathcal{L}}^{\mathcal{B}} &= \mathbf{R}_{\mathcal{L}}^{\mathcal{B}} \boldsymbol{\omega}_{\mathcal{B}/\mathcal{L}}^{\mathcal{L}} \\
\boldsymbol{\omega}_{\mathcal{L}/\mathcal{I}}^{\mathcal{B}} &= \mathbf{R}_{\mathcal{L}}^{\mathcal{B}} \boldsymbol{\omega}_{\mathcal{L}/\mathcal{I}}^{\mathcal{L}}
\end{align}

\subsection{프로젝트에서 사용하는 각속도 명명 규칙}

본 프로젝트의 코드에서는 각속도를 다음과 같은 규칙으로 표기한다:
\begin{center}
\texttt{omega\_sXeY\_observZ}
\end{center}

여기서:
\begin{itemize}
    \item \texttt{sXeY}: Frame Y relative to Frame X (X의 Y에 대한 상대 각속도)
    \item \texttt{observZ}: Observed/expressed in Frame Z (Z 프레임에서 표현)
    \item 수학적 표기: \texttt{omega\_sXeY\_observZ} = $\boldsymbol{\omega}_{\mathcal{Y}/\mathcal{X}}^{\mathcal{Z}}$
\end{itemize}

예시:
\begin{itemize}
    \item \texttt{omega\_sIeB\_observB} = $\boldsymbol{\omega}_{\mathcal{B}/\mathcal{I}}^{\mathcal{B}}$ (Deputy의 절대 각속도)
    \item \texttt{omega\_sLeA\_observA} = $\boldsymbol{\omega}_{\mathcal{A}/\mathcal{L}}^{\mathcal{A}}$ (Chief의 LVLH 대비 각속도)
    \item \texttt{omega\_sBeA\_observB} = $\boldsymbol{\omega}_{\mathcal{A}/\mathcal{B}}^{\mathcal{B}}$ (Deputy에서 본 Chief 상대 각속도)
\end{itemize}

\subsection{프로젝트에서 사용하는 상대 각속도}

\subsubsection{Deputy에서 관찰한 Chief의 각속도}

Deputy frame에서 표현된 Chief의 절대 각속도:
\begin{equation}
\boldsymbol{\omega}_{\mathcal{A}/\mathcal{I}}^{\mathcal{B}} = \mathbf{R}_{\mathcal{A}}^{\mathcal{B}} \boldsymbol{\omega}_{\mathcal{A}/\mathcal{I}}^{\mathcal{A}}
\end{equation}

Deputy에서 관찰한 Chief의 상대 각속도:
\begin{equation}
\boldsymbol{\omega}_{\mathcal{A}/\mathcal{B}}^{\mathcal{B}} = \boldsymbol{\omega}_{\mathcal{A}/\mathcal{I}}^{\mathcal{B}} - \boldsymbol{\omega}_{\mathcal{B}/\mathcal{I}}^{\mathcal{B}}
\end{equation}

또는 반대 방향:
\begin{equation}
\boldsymbol{\omega}_{\mathcal{B}/\mathcal{A}}^{\mathcal{B}} = \boldsymbol{\omega}_{\mathcal{B}/\mathcal{I}}^{\mathcal{B}} - \boldsymbol{\omega}_{\mathcal{A}/\mathcal{I}}^{\mathcal{B}}
\end{equation}

\subsection{프로젝트에서 사용되는 쿼터니언 체인}

Formation flying에서 다양한 프레임 간 쿼터니언 관계:

\begin{equation}
\mathbf{q}_{\mathcal{A}}^{\mathcal{I}} = \mathbf{q}_{\mathcal{L}}^{\mathcal{I}} \otimes \mathbf{q}_{\mathcal{A}}^{\mathcal{L}} \quad \text{(Chief의 절대 자세)}
\end{equation}

\begin{equation}
\mathbf{q}_{\mathcal{B}}^{\mathcal{I}} = \mathbf{q}_{\mathcal{L}}^{\mathcal{I}} \otimes \mathbf{q}_{\mathcal{B}}^{\mathcal{L}} \quad \text{(Deputy의 절대 자세)}
\end{equation}

\begin{equation}
\mathbf{q}_{\mathcal{B}}^{\mathcal{A}} = (\mathbf{q}_{\mathcal{A}}^{\mathcal{L}})^{-1} \otimes \mathbf{q}_{\mathcal{B}}^{\mathcal{L}} = (\mathbf{q}_{\mathcal{A}}^{\mathcal{I}})^{-1} \otimes \mathbf{q}_{\mathcal{B}}^{\mathcal{I}} \quad \text{(상대 자세)}
\end{equation}


\subsubsection{Camera Frame 관계}
카메라는 Deputy body frame에 장착되며, 다음과 같은 고정된 관계를 갖는다:
\begin{align}
\mathbf{R}_{\mathcal{B}}^{\mathcal{C}} &= \begin{bmatrix}
1 & 0 & 0 \\
0 & 0 & -1 \\
0 & 1 & 0
\end{bmatrix} \quad \text{(Body to Camera)} \\
\text{Camera axes}: & \quad \mathbf{x}_{\mathcal{C}} = \mathbf{x}_{\mathcal{B}}, \quad \mathbf{y}_{\mathcal{C}} = \mathbf{z}_{\mathcal{B}}, \quad \mathbf{z}_{\mathcal{C}} = \mathbf{y}_{\mathcal{B}}
\end{align}

여기서 $\mathbf{z}_{\mathcal{C}}$는 카메라 광축(optical axis)으로 Deputy의 $+\mathbf{y}_{\mathcal{B}}$ 방향(Chief pointing)과 일치한다.

\section{23-State Coupled EKF}

\subsection{상태 벡터 정의}
23차원 상태 벡터 $\mathbf{x}$는 다음과 같이 정의된다:
\begin{equation}
\mathbf{x} = \begin{bmatrix}
\mathbf{r}_{\mathcal{B}}^{\mathcal{L}} \\
\mathbf{v}_{\mathcal{B}}^{\mathcal{L}} \\
\delta\boldsymbol{\theta}_{\mathcal{B}/\mathcal{I}} \\
\delta\boldsymbol{\theta}_{\mathcal{B}/\mathcal{A}} \\
\boldsymbol{\omega}_{\mathcal{A}/\mathcal{I}}^{\mathcal{A}} \\
\mathbf{b}_{\text{acc}} \\
\mathbf{b}_{\text{gy}}
\end{bmatrix} \in \mathbb{R}^{21}
\end{equation}

여기서:
\begin{itemize}
    \item $\mathbf{r}_{\mathcal{B}}^{\mathcal{L}} \in \mathbb{R}^3$: Deputy의 LVLH 프레임 상대 위치
    \item $\mathbf{v}_{\mathcal{B}}^{\mathcal{L}} \in \mathbb{R}^3$: Deputy의 LVLH 프레임 상대 속도
    \item $\delta\boldsymbol{\theta}_{\mathcal{B}/\mathcal{I}} \in \mathbb{R}^3$: Deputy 절대 자세 오차 벡터
    \item $\delta\boldsymbol{\theta}_{\mathcal{B}/\mathcal{A}} \in \mathbb{R}^3$: 상대 자세 오차 벡터
    \item $\boldsymbol{\omega}_{\mathcal{A}/\mathcal{I}}^{\mathcal{A}} \in \mathbb{R}^3$: Chief의 절대 각속도 (Chief frame 표현)
    \item $\mathbf{b}_{\text{acc}} \in \mathbb{R}^3$: 가속도계 바이어스
    \item $\mathbf{b}_{\text{gy}} \in \mathbb{R}^3$: 자이로스코프 바이어스
\end{itemize}

\textbf{참고:} 실제 구현에서는 쿼터니언 $\mathbf{q}_{\mathcal{B}}^{\mathcal{I}}$와 $\mathbf{q}_{\mathcal{B}}^{\mathcal{A}}$를 저장하지만, 
EKF 오차 상태는 3차원 오차 벡터를 사용한다.

\subsection{시스템 모델}

\subsubsection{상대 궤도 동역학 (Clohessy-Wiltshire)}
LVLH 프레임에서 Deputy의 상대 운동:
\begin{equation}
\begin{aligned}
\dot{\mathbf{r}}_{\mathcal{B}}^{\mathcal{L}} &= \mathbf{v}_{\mathcal{B}}^{\mathcal{L}} \\
\dot{\mathbf{v}}_{\mathcal{B}}^{\mathcal{L}} &= \begin{bmatrix}
3n^2 r_x + 2n v_y \\
-2n v_x \\
-n^2 r_z
\end{bmatrix} + \mathbf{R}_{\mathcal{I}}^{\mathcal{L}}\mathbf{R}_{\mathcal{B}}^{\mathcal{I}}\mathbf{a}_{\mathcal{B}}^{\mathcal{B}}
\end{aligned}
\end{equation}

여기서 $\mathbf{a}_{\mathcal{B}}^{\mathcal{B}} = \mathbf{a}_{\text{meas}} - \mathbf{b}_{\text{acc}}$는 보정된 가속도 측정값이다.

\subsubsection{자세 동역학}
Deputy 절대 자세:
\begin{equation}
\dot{\mathbf{q}}_{\mathcal{B}}^{\mathcal{I}} = \frac{1}{2}\mathbf{q}_{\mathcal{B}}^{\mathcal{I}} \otimes \begin{bmatrix} 0 \\ \boldsymbol{\omega}_{\mathcal{B}/\mathcal{I}}^{\mathcal{B}} \end{bmatrix}
\end{equation}

상대 자세 (협조 케이스):
\begin{equation}
\dot{\mathbf{q}}_{\mathcal{B}}^{\mathcal{A}} = \frac{1}{2}\boldsymbol{\Theta}(\boldsymbol{\omega}_{\mathcal{A}/\mathcal{I}}^{\mathcal{A}}, \boldsymbol{\omega}_{\mathcal{B}/\mathcal{I}}^{\mathcal{B}})\mathbf{q}_{\mathcal{B}}^{\mathcal{A}}
\end{equation}

\subsubsection{Chief 각속도 모델}
Chief의 절대 각속도는 거의 상수로 모델링:
\begin{equation}
\dot{\boldsymbol{\omega}}_{\mathcal{A}/\mathcal{I}}^{\mathcal{A}} = \mathbf{w}_{\omega}
\end{equation}
여기서 $\mathbf{w}_{\omega} \sim \mathcal{N}(0, \sigma_{\omega}^2\mathbf{I}_3)$는 작은 process noise이다.

\subsubsection{센서 바이어스 모델}
일차 Gauss-Markov 프로세스:
\begin{equation}
\begin{aligned}
\dot{\mathbf{b}}_{\text{acc}} &= -\frac{1}{\tau_{\text{acc}}}\mathbf{b}_{\text{acc}} + \mathbf{w}_{\text{acc}} \\
\dot{\mathbf{b}}_{\text{gy}} &= -\frac{1}{\tau_{\text{gy}}}\mathbf{b}_{\text{gy}} + \mathbf{w}_{\text{gy}}
\end{aligned}
\end{equation}

\subsection{측정 모델}

\subsubsection{Star Tracker 측정}
절대 자세 측정:
\begin{equation}
\mathbf{z}_{\text{ST}} = 2\mathbf{q}_{\text{err}}(2:4) \approx \delta\boldsymbol{\theta}_{\mathcal{B}/\mathcal{I}}
\end{equation}
여기서 $\mathbf{q}_{\text{err}} = \mathbf{q}_{\mathcal{B}}^{\mathcal{I}} \otimes (\hat{\mathbf{q}}_{\mathcal{B}}^{\mathcal{I}})^{-1}$이다.

\subsubsection{카메라 측정}
Chief body의 특징점 $\mathbf{p}_i^{\mathcal{A}}$의 픽셀 좌표:
\begin{equation}
\mathbf{z}_{\text{cam},i} = \mathbf{h}(\mathbf{p}_i^{\mathcal{A}}, \mathbf{r}_{\mathcal{B}}^{\mathcal{L}}, \mathbf{q}_{\mathcal{B}}^{\mathcal{I}}, \mathbf{q}_{\mathcal{B}}^{\mathcal{A}})
\end{equation}

투영 과정:
\begin{equation}
\begin{aligned}
\mathbf{p}_i^{\mathcal{B}} &= \mathbf{R}_{\mathcal{A}}^{\mathcal{B}}\mathbf{p}_i^{\mathcal{A}} + \mathbf{t}_{\mathcal{A}}^{\mathcal{B}} \\
\mathbf{p}_i^{\mathcal{C}} &= \mathbf{R}_{\mathcal{B}}^{\mathcal{C}}\mathbf{p}_i^{\mathcal{B}} \\
\mathbf{z}_{\text{cam},i} &= \boldsymbol{\pi}(\mathbf{p}_i^{\mathcal{C}})
\end{aligned}
\end{equation}

여기서 $\mathbf{t}_{\mathcal{A}}^{\mathcal{B}} = -\mathbf{R}_{\mathcal{L}}^{\mathcal{B}}\mathbf{r}_{\mathcal{B}}^{\mathcal{L}}$는 Deputy에서 본 Chief 원점이다.

\subsubsection{각속도 간접 측정}
카메라 업데이트 시점에서 수치 미분을 통한 각속도 추정:
\begin{equation}
\hat{\boldsymbol{\omega}}_{\mathcal{A}/\mathcal{I}}^{\mathcal{A}} = \boldsymbol{\omega}_{\mathcal{I}/\mathcal{B}}^{\mathcal{A}} - \boldsymbol{\omega}_{\mathcal{B}/\mathcal{A}}^{\mathcal{A}}
\end{equation}

여기서:
\begin{equation}
\boldsymbol{\omega}_{\mathcal{B}/\mathcal{A}}^{\mathcal{A}} = 2\boldsymbol{\Xi}^T(\mathbf{q}_{\mathcal{B}}^{\mathcal{A}})\dot{\mathbf{q}}_{\mathcal{B}}^{\mathcal{A}}
\end{equation}

\subsection{연속시간 시스템 행렬}
선형화된 연속시간 시스템:
\begin{equation}
\dot{\mathbf{x}} = \mathbf{A}_c\mathbf{x} + \mathbf{w}
\end{equation}

$\mathbf{A}_c \in \mathbb{R}^{21 \times 21}$의 주요 블록:
\begin{equation}
\mathbf{A}_c = \begin{bmatrix}
\mathbf{0} & \mathbf{I}_3 & \mathbf{0} & \mathbf{0} & \mathbf{0} & \mathbf{0} & \mathbf{0} \\
\mathbf{A}_{21} & \mathbf{A}_{22} & \mathbf{A}_{23} & \mathbf{0} & \mathbf{0} & \mathbf{A}_{26} & \mathbf{0} \\
\mathbf{0} & \mathbf{0} & \mathbf{A}_{33} & \mathbf{0} & \mathbf{0} & \mathbf{0} & \mathbf{A}_{37} \\
\mathbf{0} & \mathbf{0} & \mathbf{0} & \mathbf{A}_{44} & \mathbf{A}_{45} & \mathbf{0} & \mathbf{A}_{47} \\
\mathbf{0} & \mathbf{0} & \mathbf{0} & \mathbf{0} & \mathbf{0} & \mathbf{0} & \mathbf{0} \\
\mathbf{0} & \mathbf{0} & \mathbf{0} & \mathbf{0} & \mathbf{0} & \mathbf{A}_{66} & \mathbf{0} \\
\mathbf{0} & \mathbf{0} & \mathbf{0} & \mathbf{0} & \mathbf{0} & \mathbf{0} & \mathbf{A}_{77}
\end{bmatrix}
\end{equation}

여기서:
\begin{itemize}
    \item $\mathbf{A}_{21} = \text{diag}(3n^2, 0, -n^2)$
    \item $\mathbf{A}_{22} = \begin{bmatrix} 0 & 2n & 0 \\ -2n & 0 & 0 \\ 0 & 0 & 0 \end{bmatrix}$
    \item $\mathbf{A}_{23} = -\mathbf{R}_{\mathcal{I}}^{\mathcal{L}}\mathbf{R}_{\mathcal{B}}^{\mathcal{I}}[\mathbf{a}_{\mathcal{B}}^{\mathcal{B}}]_{\times}$
    \item $\mathbf{A}_{33} = -[\boldsymbol{\omega}_{\mathcal{B}/\mathcal{I}}^{\mathcal{B}}]_{\times}$
    \item $\mathbf{A}_{44} = \frac{1}{2}\boldsymbol{\Theta}_{(2:4,2:4)}$
    \item $\mathbf{A}_{45} = \frac{1}{2}\boldsymbol{\Xi}_{(2:4,:)}$
\end{itemize}

\subsection{Van Loan 방법을 이용한 이산화}
State transition matrix와 process noise covariance:
\begin{equation}
\mathbf{M} = \begin{bmatrix}
-\mathbf{A}_c & \mathbf{Q}_c \\
\mathbf{0} & \mathbf{A}_c^T
\end{bmatrix}\Delta t, \quad
\exp(\mathbf{M}) = \begin{bmatrix}
\cdot & \boldsymbol{\Phi}^{-T}\mathbf{Q}_d \\
\mathbf{0} & \boldsymbol{\Phi}^T
\end{bmatrix}
\end{equation}

\subsection{측정 자코비안}
카메라 측정에 대한 자코비안:
\begin{equation}
\mathbf{H}_{\text{cam}} = \frac{\partial \mathbf{h}}{\partial \mathbf{x}} = 
[\mathbf{H}_r \quad \mathbf{0} \quad \mathbf{H}_{\theta_{\mathcal{B}/\mathcal{I}}} \quad \mathbf{H}_{\theta_{\mathcal{B}/\mathcal{A}}} \quad \mathbf{0} \quad \mathbf{0} \quad \mathbf{0}]
\end{equation}

여기서:
\begin{itemize}
    \item $\mathbf{H}_r = -\mathbf{J}_{\text{pix}}\mathbf{R}_{\mathcal{B}}^{\mathcal{C}}\mathbf{R}_{\mathcal{L}}^{\mathcal{B}}$
    \item $\mathbf{H}_{\theta_{\mathcal{B}/\mathcal{I}}} = \mathbf{J}_{\text{pix}}\mathbf{R}_{\mathcal{B}}^{\mathcal{C}}\mathbf{R}_{\mathcal{I}}^{\mathcal{B}}[\mathbf{r}_{\mathcal{B}}^{\mathcal{I}}]_{\times}$
    \item $\mathbf{H}_{\theta_{\mathcal{B}/\mathcal{A}}} = \mathbf{J}_{\text{pix}}\mathbf{R}_{\mathcal{B}}^{\mathcal{C}}[\mathbf{p}_i^{\mathcal{B}}]_{\times}$
\end{itemize}
\section{비협조 상대 자세 동역학}

\subsection{상대 자세 동역학 비교}

\subsubsection{협조 케이스 (Cooperative Case)}
두 위성이 서로의 절대 각속도를 공유하는 경우:
\begin{equation}
\dot{\mathbf{q}}_{\mathcal{B}}^{\mathcal{A}} = \frac{1}{2}\boldsymbol{\Theta}(\boldsymbol{\omega}_{\mathcal{A}/\mathcal{I}}^{\mathcal{A}}, \boldsymbol{\omega}_{\mathcal{B}/\mathcal{I}}^{\mathcal{B}})\mathbf{q}_{\mathcal{B}}^{\mathcal{A}}
\end{equation}

여기서:
\begin{equation}
\boldsymbol{\Theta}(\boldsymbol{\omega}_{\mathcal{A}/\mathcal{I}}^{\mathcal{A}}, \boldsymbol{\omega}_{\mathcal{B}/\mathcal{I}}^{\mathcal{B}}) = \boldsymbol{\Omega}(\boldsymbol{\omega}_{\mathcal{B}/\mathcal{I}}^{\mathcal{B}}) - \boldsymbol{\Gamma}(\boldsymbol{\omega}_{\mathcal{A}/\mathcal{I}}^{\mathcal{A}})
\end{equation}

\subsubsection{비협조 케이스 (Non-cooperative Case)}
Chief의 각속도 정보 없이 상대 각속도만을 사용:
\begin{equation}
\dot{\mathbf{q}}_{\mathcal{B}}^{\mathcal{A}} = \frac{1}{2}\mathbf{q}_{\mathcal{B}}^{\mathcal{A}} \otimes \begin{bmatrix} 0 \\ \boldsymbol{\omega}_{\mathcal{B}/\mathcal{A}}^{\mathcal{B}} \end{bmatrix} = \frac{1}{2}\boldsymbol{\Omega}(\boldsymbol{\omega}_{\mathcal{B}/\mathcal{A}}^{\mathcal{B}})\mathbf{q}_{\mathcal{B}}^{\mathcal{A}}
\end{equation}

상대 각속도는 다음과 같이 계산:
\begin{equation}
\boldsymbol{\omega}_{\mathcal{B}/\mathcal{A}}^{\mathcal{B}} = \boldsymbol{\omega}_{\mathcal{B}/\mathcal{I}}^{\mathcal{B}} - \mathbf{R}_{\mathcal{A}}^{\mathcal{B}}\boldsymbol{\omega}_{\mathcal{A}/\mathcal{I}}^{\mathcal{A}}
\end{equation}

\subsection{상태 벡터 재정의}

\subsubsection{Option 1: 상대 각속도 직접 추정 (21-State)}
Chief 각속도 대신 상대 각속도를 상태에 포함:
\begin{equation}
\mathbf{x}_{\text{rel}} = \begin{bmatrix}
\mathbf{r}_{\mathcal{B}}^{\mathcal{L}} \\
\mathbf{v}_{\mathcal{B}}^{\mathcal{L}} \\
\delta\boldsymbol{\theta}_{\mathcal{B}/\mathcal{I}} \\
\delta\boldsymbol{\theta}_{\mathcal{B}/\mathcal{A}} \\
\boldsymbol{\omega}_{\mathcal{B}/\mathcal{A}}^{\mathcal{B}} \\
\mathbf{b}_{\text{acc}} \\
\mathbf{b}_{\text{gy}}
\end{bmatrix} \in \mathbb{R}^{21}
\end{equation}

상대 각속도 모델:
\begin{equation}
\dot{\boldsymbol{\omega}}_{\mathcal{B}/\mathcal{A}}^{\mathcal{B}} = \mathbf{w}_{\omega,\text{rel}}
\end{equation}

\subsubsection{Option 2: 축소 상태 벡터 (17-State)}
Chief 각속도를 제거하고 상대 자세만 전파:
\begin{equation}
\mathbf{x}_{\text{reduced}} = \begin{bmatrix}
\mathbf{r}_{\mathcal{B}}^{\mathcal{L}} \\
\mathbf{v}_{\mathcal{B}}^{\mathcal{L}} \\
\delta\boldsymbol{\theta}_{\mathcal{B}/\mathcal{I}} \\
\delta\boldsymbol{\theta}_{\mathcal{B}/\mathcal{A}} \\
\mathbf{b}_{\text{acc}} \\
\mathbf{b}_{\text{gy}}
\end{bmatrix} \in \mathbb{R}^{17}
\end{equation}

\subsection{선형화된 시스템 행렬 변경}

\subsubsection{Option 1: 상대 각속도 추정 시스템}

연속시간 시스템 행렬 $\mathbf{A}_{\text{rel}} \in \mathbb{R}^{21 \times 21}$의 상대 자세 블록:

\begin{equation}
\mathbf{A}_{\text{rel}}(10:12, :) = 
\begin{bmatrix}
\mathbf{0}_{3\times9} & \frac{1}{2}\boldsymbol{\Omega}_{(2:4,2:4)}(\boldsymbol{\omega}_{\mathcal{B}/\mathcal{A}}^{\mathcal{B}}) & \frac{1}{2}\boldsymbol{\Psi}_{(2:4,:)}(\mathbf{q}_{\mathcal{B}}^{\mathcal{A}}) & \mathbf{0}_{3\times3} & -\frac{1}{2}\boldsymbol{\Psi}_{(2:4,:)}(\mathbf{q}_{\mathcal{B}}^{\mathcal{A}})
\end{bmatrix}
\end{equation}

Process noise covariance에 상대 각속도 항 추가:
\begin{equation}
\mathbf{Q}_{\text{rel}}(13:15, 13:15) = \sigma_{\omega,\text{rel}}^2 \mathbf{I}_3
\end{equation}

\subsubsection{Option 2: 축소 시스템}

연속시간 시스템 행렬 $\mathbf{A}_{\text{reduced}} \in \mathbb{R}^{17 \times 17}$:

상대 자세는 측정값으로만 업데이트되므로:
\begin{equation}
\mathbf{A}_{\text{reduced}}(10:12, :) = \mathbf{0}_{3\times17}
\end{equation}

또는 상대 각속도를 근사적으로 계산:
\begin{equation}
\hat{\boldsymbol{\omega}}_{\mathcal{B}/\mathcal{A}}^{\mathcal{B}} \approx \boldsymbol{\omega}_{\mathcal{B}/\mathcal{I}}^{\mathcal{B}} - \mathbf{R}_{\mathcal{A}}^{\mathcal{B}}\hat{\boldsymbol{\omega}}_{\mathcal{A}/\mathcal{I}}^{\mathcal{A}}
\end{equation}

여기서 $\hat{\boldsymbol{\omega}}_{\mathcal{A}/\mathcal{I}}^{\mathcal{A}}$는 사전 지식 또는 예측 모델 기반.

\subsection{측정 모델 수정}

\subsubsection{상대 각속도 측정 (Option 1)}

카메라 업데이트 시점에서 수치 미분:
\begin{equation}
\hat{\boldsymbol{\omega}}_{\mathcal{B}/\mathcal{A}}^{\mathcal{B}} = 2\boldsymbol{\Xi}^T(\mathbf{q}_{\mathcal{B},k-1}^{\mathcal{A}})\frac{\mathbf{q}_{\mathcal{B},k}^{\mathcal{A}} - \mathbf{q}_{\mathcal{B},k-1}^{\mathcal{A}}}{\Delta t}
\end{equation}

측정 방정식:
\begin{equation}
\mathbf{z}_{\omega,\text{rel}} = \hat{\boldsymbol{\omega}}_{\mathcal{B}/\mathcal{A}}^{\mathcal{B}} - \boldsymbol{\omega}_{\mathcal{B}/\mathcal{A}}^{\mathcal{B}}
\end{equation}

측정 자코비안:
\begin{equation}
\mathbf{H}_{\omega,\text{rel}} = [\mathbf{0}_{3\times12} \quad \mathbf{I}_3 \quad \mathbf{0}_{3\times6}]
\end{equation}

\subsubsection{상대 자세 직접 측정 (Option 2)}

Vision 기반 상대 자세 추정:
\begin{equation}
\hat{\mathbf{q}}_{\mathcal{B}}^{\mathcal{A}} = \text{MLPnP}(\{\mathbf{p}_i^{\mathcal{A}}, \mathbf{u}_i\})
\end{equation}

측정 오차:
\begin{equation}
\mathbf{z}_{\text{att,rel}} = 2\mathbf{q}_{\text{err,rel}}(2:4)
\end{equation}

여기서 $\mathbf{q}_{\text{err,rel}} = \hat{\mathbf{q}}_{\mathcal{B}}^{\mathcal{A}} \otimes (\mathbf{q}_{\mathcal{B}}^{\mathcal{A}})^{-1}$

\subsection{구현 상세}

\subsubsection{상태 전파 코드 변경}
협조 케이스:
\begin{verbatim}
% Cooperative case
Theta = GetThetaMatrix(omega_AI_A, omega_BI_B);
q_dot_BA = 0.5 * Theta * q_B2A_prev;
\end{verbatim}

비협조 케이스:
\begin{verbatim}
% Non-cooperative case (Option 1)
omega_BA_B = omega_BI_B - R_A2B * omega_AI_A_assumed;
Omega_BA = GetOmegaMatrix([0; omega_BA_B]);
q_dot_BA = 0.5 * Omega_BA * q_B2A_prev;
\end{verbatim}

\subsubsection{자코비안 계산 변경}
협조 케이스 자코비안:
\begin{equation}
\frac{\partial \dot{\mathbf{q}}_{\mathcal{B}}^{\mathcal{A}}}{\partial \boldsymbol{\omega}_{\mathcal{A}/\mathcal{I}}^{\mathcal{A}}} = \frac{1}{2}\boldsymbol{\Xi}(\mathbf{q}_{\mathcal{B}}^{\mathcal{A}})
\end{equation}

비협조 케이스 자코비안:
\begin{equation}
\frac{\partial \dot{\mathbf{q}}_{\mathcal{B}}^{\mathcal{A}}}{\partial \boldsymbol{\omega}_{\mathcal{B}/\mathcal{A}}^{\mathcal{B}}} = \frac{1}{2}\boldsymbol{\Psi}(\mathbf{q}_{\mathcal{B}}^{\mathcal{A}})
\end{equation}

\subsection{성능 비교}

\begin{table}[h]
\centering
\begin{tabular}{lcc}
\toprule
\textbf{특성} & \textbf{협조 케이스} & \textbf{비협조 케이스} \\
\midrule
상태 차원 & 21 (또는 23) & 17 또는 21 \\
통신 요구사항 & 필요 & 불필요 \\
상대 자세 정확도 & 높음 & 중간 \\
계산 복잡도 & 높음 & 낮음 \\
수렴 속도 & 빠름 & 느림 \\
\bottomrule
\end{tabular}
\caption{협조/비협조 케이스 비교}
\end{table}

\subsection{선택 기준}

\begin{itemize}
\item \textbf{협조 케이스 사용}: 
    \begin{itemize}
    \item Chief-Deputy 간 통신 가능
    \item 높은 상대 자세 정확도 요구
    \item Formation의 tight control 필요
    \end{itemize}
    
\item \textbf{비협조 케이스 사용}:
    \begin{itemize}
    \item 통신 불가 또는 제한적
    \item Chief가 비협조적 타겟
    \item 자율성 중시
    \end{itemize}
\end{itemize}

\end{document}
